\documentclass[a4paper]{article}
\usepackage{amsmath,amssymb,latexsym,graphicx,csquotes}

\setlength{\parindent}{0cm} \setlength{\parskip}{0cm} \sloppy
\markright{chapter}


\begin{document}

\begin{titlepage}
\setlength{\topmargin}{8cm}
\begin{center}
 \Huge NOMMA 1.0\\
 \vspace{1cm}
 \Large Non-Parametric Optimization Methods\\
		for Model Assessment\\
 \vspace{1cm}
 \normalsize Volkmar Sauerland\\
 Kiel University\\
 \vspace{1cm}
 \Large December 2017 \\
\end{center}
\end{titlepage}

\section{\label{intro}Introduction}
NOMMA 1.0 provides C++ implementations
of a couple of non-parametric regression methods.
The regression methods are of interest on their own
but also useful for the assessment of parametric models~\cite{SLL+17}
since they can provide lower error bounds on them.\par
There are two subfolders \texttt{regression} and \texttt{regressionCPX}.
The first folder contains implementations of tailored algorithms
for certain non-parametric regression types~\cite{BBBB72,DP91,YW09}.
These implementations do not require third party software to be installed.
The folder \texttt{regressionCPX} contains an extension of the content
in \texttt{regression}.
The additional methods rely on direct formulations of regression problems
in terms of \emph{quadratic programs (QPs)}.
The QPs are build and solved using the third party software \emph{CPLEX},
which must be installed on your system in order to use \texttt{regressionCPX}.\par
\section{\label{usage}Usage}
\subsection{Generating and Running Example Program}
Both subfolders \texttt{} and \texttt{} contain the following files
\begin{itemize}
\item \texttt{Makefile}. On a LINUX platform with GNU g++ compiler,
you can generate the executable \texttt{regEx} by typing \texttt{make}.
\item \texttt{regEx.cpp}. The example main program reads
observational data (an uni-variate time series)
and applies some regression methods implemented in \texttt{regression.cpp}
and writes obtained regression time series to output files.
Each line of the input file (here \texttt{sineLikeNoise\_0.45\_200.dat})
consist of two real numbers: a time value and an associated measurement value.
The first column (the time values) must be sorted in increasing order.
The output files have the same format as the input file.
\item \texttt{regression.cpp/hpp}. Implements regression methods
that are considered in our article~\cite{SLL+17}
for the assessment of biogeochemical ocean models.
\item \texttt{knot.cpp/hpp}. An auxiliary structure
introduced in~\cite{YW09} and used to calculate
\enquote{isotonic regression under Lipschitz constraint}.
\item \texttt{sineLikeNoise\_0.45\_200.dat}. Data file with 200 samples
of the function $\sin(t)+0.3\cdot \sin(t)+\mathcal{N}(0,0.45)$,
serving as test observational data
\end{itemize}
After having generated the executable \texttt{regEx}
(by typing \texttt{make}) you can type,
e.g., \texttt{./regEx} to calculate
some non-parametric regression time-series
for the example data \texttt{sineLikeNoise\_0.45\_200.dat}.


\subsection{Usage for Own Applications}



REFERENCES
----------

[BBBB72] Barlow, R. E., Bartholomew, D. J., Bremmer, J. M., Brunk, H. D.
         *Statistical Inference under order restrictions. The theory and
         applications of isotonic regression*. John Wiley \& Sons, 1972

[DP91]   I. C. Demetriou. M. J. D. Powell.
         *Least Squares Smoothing of Univariate Data to achieve Piecewise
         Monotonicity*.
         IMA Journal of Numerical Analysis 11:411-432 (1991)
         doi: 10.1093/imanum/11.3.411

[YW09]   Yeganova, L., Wilbur, W. J.
         *Isotonic Regression under Lipschitz Constraint*.
         J. Optim. Theory Allp. 141:429-443 (2009)
         doi: 10.1007/s10957-008-9477-0

[SLL+17] Sauerland, V., Loeptien, U., Leonhard, C., Oschlies, A., Srivastav, A.
         *Error assessment of biogeochemical models by lower bound methods*.
         Geoscientific Model Development Discussions (2017)


\end{document}
